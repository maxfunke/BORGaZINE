\documentclass[twoside, a5paper, 9pt, openright]{memoir}
% \documentclass[a5paper, 8pt]{book}
% \usepackage[margin=1.5cm, footskip=-1cm]{geometry}
% \setmainfont{SourceSansPro}[%
% %    Path= /usr/local/texlive/2014/texmf-dist/fonts/opentype/adobe/sourcesanspro/ ,
%     Extension = .otf ,
%     UprightFont = *-Regular ,
%     ItalicFont = *-RegularIt ,
%     BoldFont = *-Bold ,
%     BoldItalicFont = *-BoldIt ]
\usepackage[%
    a5paper,
%   includeheadfoot,
    head=\baselineskip,  % distance from bottom of header to block of text aka \headsep e.g. \baselineskip
    foot=1cm,  % distance from top of footer to block of text aka \footskip
    headheight=12pt,     % height for the header block (no equivalent for footer)
%   heightrounded,       % ensure an integer number of lines
    marginparwidth=2cm,  % right marginal note width
    marginparsep=2mm,    % distance from text block to marginal note box
%   height=\textheight,  % height of the text block
%   width=\textwidth,    % width of the text block
    top=2.0cm,           % distance of the text block from the top of the page
    bottom=1.5cm,
    left=1.5cm,
    right=1.5cm,
%    showframe,           % show the main blocks
%    verbose,             % show the values of the parameters in the log file
]{geometry}
\usepackage{xcolor}
\usepackage{multicol}
\usepackage{blindtext}
\usepackage{tikz}
\usetikzlibrary{calc, trees, positioning,arrows.meta, chains, shapes.geometric, decorations.pathreplacing, decorations.pathmorphing, shapes, matrix, shapes.symbols}
\tikzset{>={Latex[width=2mm, length=1.2mm]}}


\usepackage{yfonts}
\usepackage{concmath}
\usepackage[T1]{fontenc}
\renewcommand*\familydefault{\ttdefault} %% Only if the base font of the document is to be typewriter style


\usepackage{titlesec}
\titleformat{\chapter}[display]
    {\normalfont\frakfamily\Large\bfseries}{}{0em}{\Huge}
% \titleformat{name=\chapter,numberless}[display]
% 	{\normalfont\LARGE\bfseries\filcenter}{}{1ex}
% 	{\vspace{2ex}}[\vspace{1ex}]
\titleformat{name=\chapter,numberless}[display]
	{\normalfont\frakfamily\Large\bfseries}{}{0em}
	{\Large}
\titleformat{\section}
    {\normalfont\frakfamily\Large\bfseries}{}{0em}{}
\titleformat{\subsection}
    {}{}{0em}{}

\usepackage{fancyhdr}

\pagestyle{fancy} % This must be here, because defaults are set and renewcommand for section marks will work.
\renewcommand{\chaptermark}[1]{%
    \markboth{#1}{}%
}
\renewcommand{\sectionmark}[1]{\markright{#1}}
\renewcommand{\subsectionmark}[1]{}

%Fancyhdr Styles
\fancypagestyle{frontmatter}{%
	\fancyhf{} % clear all fields
	\renewcommand{\headrulewidth}{0pt}
	\lhead{}
	\lfoot{}
	\cfoot{}
	\rfoot{}
}%
\fancypagestyle{mainmatter}{%
	\fancyhf{} % clear all fields
	\renewcommand{\headrulewidth}{0.3pt}
	\renewcommand{\footrulewidth}{0.0pt}
	% \fancyhead[LE,RO]{\leftmark}
	\fancyhead[LE,RO]{\leftmark}
	\fancyhead[LO,RE]{MÖRK:AZINE}
	\fancyfoot[LE,RO]{\thepage}
}%

\fancypagestyle{IHA-fancy-style}{%
\fancyhf{}% Clear header and footer
\renewcommand{\headrulewidth}{0.3pt}% Line at the header visible
\renewcommand{\footrulewidth}{0.0pt}
\fancyhead[LE,RO]{\rightmark}
\fancyhead[LO,RE]{MÖRK:AZINE}
\fancyfoot[LE,RO]{\thepage}% Custom footer
}
% \pagestyle{IHA-fancy-style}

% Redefine the plain page style
\fancypagestyle{plain}{%
\renewcommand{\headrulewidth}{0.0pt}% Line at the header invisible
\renewcommand{\footrulewidth}{0.0pt}
	\fancyhf{}%
	\fancyfoot[C]{\thepage}%
}

\newcommand{\DTwo}[2]{\colorbox{purple}{\textbf{#1(d2)#2}}}
\newcommand{\DFour}[2]{\colorbox{orange}{\textbf{#1(d4)#2}}}
\newcommand{\DSix}[2]{\colorbox{green}{\textbf{#1(d6)#2}}}
\newcommand{\DEight}[2]{\colorbox{magenta}{\textbf{#1(d8)#2}}}
\newcommand{\DTen}[2]{\colorbox{cyan}{\textbf{#1(d10)#2}}}
\newcommand{\DTwelve}[2]{\colorbox{yellow}{\textbf{#1(d12)#2}}}
\newcommand{\DTwenty}[2]{\colorbox{lightgray}{\textbf{#1(d20)#2}}}
\newcommand{\DOneHundret}[2]{\colorbox{cyan}{\textbf{#1(d100)#2}}}



% Edit inside the { brackets } to change these.

\title{MÖRK:AZINE}
\author{Fax Mumpe}
\date{April 2025}

\begin{document}
\pagestyle{frontmatter}
\maketitle
\newpage
\pagestyle{mainmatter}
\chapter{Kapitel}
\section*{sec1}
\addcontentsline{toc}{section}{sec1} 
\newpage
\section*{sec2}
\addcontentsline{toc}{section}{sec2} 
text \textbf{bold text} text.
\DTwo{2}{+2}
\DFour{2}{+2}
\DSix{2}{+2}
\DEight{}{}
\DTen{}{}
\DTwelve{}{}
\DTwenty{}{}
\DOneHundret{roll a }{}
\subsection*{subsection}
text \emph{emphasized text} text.
discovered the structure of DNA.
A table:
\begin{table}[!th]
	\begin{tabular}{|l|c|r|}
		\hline
		first  & row & data \\
		second & row & data \\
		\hline
	\end{tabular}
	\caption{This is the caption}
	\label{ex:table}
\end{table}

\newpage

\begin{figure}[hbt]
	\centering
	\tikzstyle{block} = [draw, text width=10em, text centered, minimum height=2em]
	\begin{tikzpicture}
		[node distance=1.35cm,
			start chain=going below,]
		\node (n1) at (0,0) [block]  {General framwork};
		\node (n2) [block, below of=n1] {Literature study};
		\node (n3) [block, below of=n2] {Data-analysis};
		\node (n4) [block, below of=n3] {Model set-up (SB \& NH)};
		\node (n5) [block, below of=n4] {Simulations};
		\node (n6) [block, below of=n5] {Sensitivity analysis};
		\node (n7) [block, below of=n6] {Model update};
		\node (n8) [block, below of=n7] {Conclusions and recommendations};
		% Connectors
		\draw [->] (n1) -- (n2);
		\draw [->] (n2) -- (n3);
		\draw [->] (n3) -- (n4);
		\draw [->] (n4) -- (n5);
		\draw [->] (n5) -- (n6);
		\draw [->] (n6) -- (n7);
		\draw [->] (n6) -- (n7);
		\draw [->] (n7) -- (n8);
		\draw [->] (n7.east) -| ++(1,0) |- (n6.east);
		\draw [->] (n7.west) -| ++(-1,0) |- (n5.west);
	\end{tikzpicture}
	
	\caption{Some title}
	\label{fig:title}
\end{figure}

\chapter{Roads to Damnation}
TRAVEL ACROSS THE DYING WORLD\\
Text Svante Landgraf Proofreading Esh
\onecolumn
\setlength{\columnsep}{0.5cm}
\setlength{\columnseprule}{0.4pt}
\begin{multicols}{2}
	\section*{Travel checklist}
	\begin{itemize}
		\item Keep track of food and water.
		\item Remember to roll for \textbf{The Calendar of Nechrubel} daily.
		\item Roll for \textbf{weather}, rerolling if it becomes samey or travel events dictate.
		\item \textbf{Information within parentheses} are things which are not clear at first glance.
	\end{itemize}

	\section*{what's the road like? \DEight{}{}}
	\begin{enumerate}
		\item[1] Almost-forgotten \textbf{dirt track}.
		\item[2] asdf.
		\item[3] brown
		\item[4-5] dingens
		\item[6-7] red
		\item[8] green
	\end{enumerate}
\end{multicols}
\hrule
\hfill \break
\hfill \break
\hfill \break
\begin{multicols}{2}
	\section{Events by the road. \DTwenty{Roll daily}{:}}
	\begin{enumerate}
		\item[1-3] Almost-forgotten \textbf{dirt track}.
		\item[4] asdf.
		\item[5-6] brown
		\item[7-8] dingens
		\item[9] red
		\item[10] A few mercenaries and their \DEight{}{ guard}. (All infected by a brain parasite.)
		\item[11] A few mercenaries and their \DEight{}{ guard}. (All infected by a brain parasite.)
		\item[12] A few mercenaries and their \DEight{}{ guard}. (All infected by a brain parasite.)
		\item[13] A few mercenaries and their \DEight{}{ guard}. (All infected by a brain parasite.)
		\item[14] \DSix{}{+1 slavers,} leading \DSix{2}{ slaves}, half beaten to death, half freshly caught.
		\item[15] A few mercenaries and their \DEight{}{ guard}. (All infected by a brain parasite.)
		\item[16] A few mercenaries and their \DEight{}{ guard}. (All infected by a brain parasite.)
		\item[17] A few mercenaries and their \DEight{}{ guard}. (All infected by a brain parasite.)
		\item[18] A few mercenaries and their \DEight{}{ guard}. (All infected by a brain parasite.)
		\item[19] A few mercenaries and their \DEight{}{ guard}. (All infected by a brain parasite.)
		\item[20] A few mercenaries and their \DEight{}{ guard}. (All infected by a brain parasite.)
	\end{enumerate}
\end{multicols}
\clearpage

\blindtext[2]

% Back cover

\newpage

\Large That's it! Have fun.

\tableofcontents

\end{document}